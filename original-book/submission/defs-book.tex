%
% Macros for the OCaml book.
%
\newcommand\note[1]{}
\newcommand\misspelled[1]{#1}

\newcommand\OCaml{OCaml}

%
% Basic macros
%
\newcommand\mr[1]{\textrm{#1}}
\newcommand\ms[1]{\textit{#1}}
\newcommand\mt[1]{\texttt{#1}}
\newcommand\mb[1]{\textbf{#1}}
\newcommand\vbar{\mathrel{|}}
\newcommand\ttemph[1]{\texttt{\emph{#1}}}
\newcommand\nt[1]{\ms{\rmfamily #1}}

\newcommand\subtype{\mathrel{\hbox{<:}}}

\newcommand\pipe{|}

%
% An indented paragraph.
%
% \renewenvironment{indent}
%     {\begin{center}
%      \begin{minipage}{\innerwidth}}
%     {\end{minipage}
%      \end{center}}

%
% Comments
%
\newif\ifcomment
\commenttrue
\newdimen\columnwidth
\setlength\columnwidth{\textwidth}
\newdimen\commentwidth
\setlength{\commentwidth}{\columnwidth}
\newcommand\comment[1]{%
  \ifcomment
    \begin{center}
      \advance\commentwidth by -0.5in
      \fbox{\begin{minipage}{\commentwidth}
            \sf #1
            \end{minipage}}
    \end{center}
  \fi}
\newcommand\nocomment[1]{}

%
% Marginal comments.
%
% Commenting symbol
%
\newcommand\commentsym{%
\begin{picture}(9, 7)
\put(4, 3.5){\circle{6}}
\put(1, 3.5){\line(1,0){6}}
\end{picture}}
%
% Comment in the margin.
%
\newcommand\mcomment[1]{%
   \ifcomment
      \commentsym\marginpar{\commentsym \em #1}
   \fi}

%
% Comment in a footnote
%
\newcommand\fcomment[1]{%
   \ifcomment
      \footnote{\comment{#1}}
   \fi}

%%%%%%%%%%%%%%%%%%%%%%%%%%%%%%%%%%%%%%%%%%%%%%%%%%%%%%%%%%%%%%%%%%%%%%%%
% Display
%

%
% Use verbatim environments for the toploop.
%
\DefineVerbatimEnvironment{topoutput}{BVerbatim}{fontshape=it}
\DefineVerbatimEnvironment{toperror}{BVerbatim}{fontshape=it}

%%%%%%%%%%%%%%%%%%%%%%%%%%%%%%%%%%%%%%%%%%%%%%%%%%%%%%%%%%%%%%%%%%%%%%%%
% Exercise section.
%
% \answers       : set if we are printing the answers
%
\newif\ifanswers
\answersfalse

\newtheorem{doexercise}{Exercise}
\newenvironment{exercise}[1]
{%
\begin{doexercise}
   \label{exercise:#1}
   \ifanswers
   \else
      \rm
   \fi
}
{\end{doexercise}}
\renewcommand\thedoexercise{\thechapter.\arabic{doexercise}}

%
% The box is just for comments.  If we don't want the answers,
% put them into a box and forget about it.
%
\newenvironment{answer}
{\rm}
{}

\newcommand\exercises{%
\section{Exercises}
\setcounter{doexercise}{0}}

\newcommand\answers{%
\answerstrue
\chapter{Answers to exercises}
\renewcommand\thedoexercise{\thesection.\arabic{doexercise}}}

%%%%%%%%%%%%%%%%%%%%%%%%%%%%%%%%%%%%%%%%%%%%%%%%%%%%%%%%%%%%%%%%%%%%%%%%
% Figures
%
\newcommand\refchapter[1]{\ref{chapter:#1}}
\newcommand\refsection[1]{\ref{section:#1}}

\newcommand\makefigure[3]{%
   \begin{figure}[#2]
   #3
   \label{figure:#1}
   \end{figure}}

%
% XXX: Hypertext target
%
\newcommand\xline[1]{#1}
\newcommand\target[2]{#2}

% -*-
% Local Variables:
% Mode: LaTeX
% fill-column: 100
% TeX-master: "paper"
% TeX-command-default: "LaTeX/dvips Interactive"
% End:
% -*-
